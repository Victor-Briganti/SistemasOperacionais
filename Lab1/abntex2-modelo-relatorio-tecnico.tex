%% abtex2-modelo-relatorio-tecnico.tex, v-1.7.1 laurocesar
%% Copyright 2012-2013 by abnTeX2 group at http://abntex2.googlecode.com/ 
%%
%% This work may be distributed and/or modified under the
%% conditions of the LaTeX Project Public License, either version 1.3
%% of this license or (at your option) any later version.
%% The latest version of this license is in
%%   http://www.latex-project.org/lppl.txt
%% and version 1.3 or later is part of all distributions of LaTeX
%% version 2005/12/01 or later.
%%
%% This work has the LPPL maintenance status `maintained'.
%% 
%% The Current Maintainer of this work is the abnTeX2 team, led
%% by Lauro César Araujo. Further information are available on 
%% http://abntex2.googlecode.com/
%%
%% This work consists of the files abntex2-modelo-relatorio-tecnico.tex,
%% abntex2-modelo-include-comandos and abntex2-modelo-references.bib
%%

% ------------------------------------------------------------------------
% ------------------------------------------------------------------------
% abnTeX2: Modelo de Relatório Técnico/Acadêmico em conformidade com 
% ABNT NBR 10719:2011 Informação e documentação - Relatório técnico e/ou
% científico - Apresentação
% ------------------------------------------------------------------------ 
% ------------------------------------------------------------------------

% Alterado por Rodrigo Campiolo para apresentação de relatórios na disciplina
% de Redes de Computadores II do Bacharelado em Ciência da Computação da UTFPR-CM.


\documentclass[
	% -- opções da classe memoir --
	12pt,				% tamanho da fonte
	%openright,			% capítulos começam em pág ímpar (insere página vazia caso preciso)
	oneside,   	        % para impressão em verso e anverso use twoside. Oposto a oneside
	a4paper,			% tamanho do papel. 
	% -- opções da classe abntex2 --
	%chapter=TITLE,		% títulos de capítulos convertidos em letras maiúsculas
	%section=TITLE,		% títulos de seções convertidos em letras maiúsculas
	%subsection=TITLE,	% títulos de subseções convertidos em letras maiúsculas
	%subsubsection=TITLE,% títulos de subsubseções convertidos em letras maiúsculas
	% -- opções do pacote babel --
	english,			% idioma adicional para hifenização
	french,				% idioma adicional para hifenização
	spanish,			% idioma adicional para hifenização
	brazil,				% o último idioma é o principal do documento
	]{pacotes/abntex2}


% ---
% PACOTES
% ---

% ---
% Pacotes fundamentais 
% ---
\usepackage{cmap}				% Mapear caracteres especiais no PDF
\usepackage{lmodern}			% Usa a fonte Latin Modern
\usepackage[T1]{fontenc}		% Selecao de codigos de fonte.
\usepackage[utf8]{inputenc}		% Codificacao do documento (conversão automática dos acentos)
\usepackage{indentfirst}		% Indenta o primeiro parágrafo de cada seção.
\usepackage{color}				% Controle das cores
\usepackage{graphicx}			% Inclusão de gráficos
% ---

% ---
% Pacotes adicionais, usados no anexo do modelo de folha de identificação
% ---
\usepackage{multicol}
\usepackage{multirow}
% ---
	
% ---
% Pacotes adicionais, usados apenas no âmbito do Modelo Canônico do abnteX2
% ---
\usepackage{lipsum}				% para geração de dummy text
% ---

% ---
% Pacotes de citações
% ---
\usepackage[brazilian,hyperpageref]{backref}	 % Paginas com as citações na bibl
\usepackage[alf]{pacotes/abntex2cite}	% Citações padrão ABNT
\usepackage{comment}
% ---

% ---
% Meus pacotes
% ---
\usepackage{float}
% ---

% --- 
% CONFIGURAÇÕES DE PACOTES
% --- 

% ---
% Configurações do pacote backref
% Usado sem a opção hyperpageref de backref
\renewcommand{\backrefpagesname}{Citado na(s) página(s):~}
% Texto padrão antes do número das páginas
\renewcommand{\backref}{}
% Define os textos da citação
\renewcommand*{\backrefalt}[4]{
	\ifcase #1 %
		Nenhuma citação no texto.%
	\or
		Citado na página #2.%
	\else
		Citado #1 vezes nas páginas #2.%
	\fi}%
% ---

% ---
% Informações de dados para CAPA e FOLHA DE ROSTO
% ---
\titulo{Instalar o GNU/Linux e compilar o núcleo}
\autor{Hendrick Felipe Scheifer\\João Victor Briganti\\Luiz Gustavo Takeda}
\local{Campo Mourão}
\data{Outubro / 2024}
\instituicao{%
  Universidade Tecnológica Federal do Paraná -- UTFPR
  \par
  Departa           mento Acadêmico de Computação -- DACOM
  \par
  Bacharelado em Ciência da Computação -- BCC
}
\tipotrabalho{Relatório técnico}
% O preambulo deve conter o tipo do trabalho, o objetivo, 
% o nome da instituição e a área de concentração 
\preambulo{Relatório técnico de atividade prática solicitado pelo professor Rodrigo Campiolo na disciplina de Sistemas Operacionais do Bacharelado em Ciência da Computação da Universidade Tecnológica Federal do Paraná.}
% ---

% ---
% Configurações de aparência do PDF final

% alterando o aspecto da cor azul
\definecolor{blue}{RGB}{41,5,195}

% informações do PDF
\makeatletter
\hypersetup{
     	%pagebackref=true,
		pdftitle={\@title}, 
		pdfauthor={\@author},
    	pdfsubject={\imprimirpreambulo},
	    pdfcreator={LaTeX with abnTeX2},
		pdfkeywords={abnt}{latex}{abntex}{abntex2}{relatório técnico}, 
		colorlinks=true,       		% false: boxed links; true: colored links
    	linkcolor=blue,          	% color of internal links
    	citecolor=blue,        		% color of links to bibliography
    	filecolor=magenta,      		% color of file links
		urlcolor=blue,
		bookmarksdepth=4
}
\makeatother
% --- 

% --- 
% Espaçamentos entre linhas e parágrafos 
% --- 

% O tamanho do parágrafo é dado por:
\setlength{\parindent}{1.3cm}

% Controle do espaçamento entre um parágrafo e outro:
\setlength{\parskip}{0.2cm}  % tente também \onelineskip

% ---
% compila o indice
% ---
\makeindex
% ---

% Omite a numeração de capítulos
\renewcommand*\thesection{\arabic{section}}



% ----
% Início do documento
% ----
\begin{document}

% Retira espaço extra obsoleto entre as frases.
\frenchspacing 

% ----------------------------------------------------------
% ELEMENTOS PRÉ-TEXTUAIS
% ----------------------------------------------------------
% \pretextual

% ---
% Capa
% ---
%\imprimircapa
% ---

% ---
% Folha de rosto
% (o * indica que haverá a ficha bibliográfica)
% ---
\imprimirfolhaderosto
% ---


% ---
% RESUMO
% ---

% resumo na língua vernácula (obrigatório)
\begin{resumo}
Este trabalho apresenta os procedimentos necessários para a instalação de uma distribuição GNU/Linux, focando na configuração e compilação do núcleo do sistema operacional. Utilizou-se o hipervisor VirtualBox (versão 7.1.2) para criar uma máquina virtual, na qual foi instalada a distribuição Debian (versão $12.7$) e o núcleo Linux (versão $6.10.11$). Durante o processo, foram executados diversos comandos básicos, possibilitando uma compreensão aprofundada da estrutura do sistema e a aplicação prática dos conceitos aprendidos em sala de aula. Os resultados evidenciam a importância da prática na consolidação do conhecimento teórico.

 \vspace{\onelineskip}
    
 \noindent
 \textbf{Palavras-chave}: núcleo. GNU/Linux. Debian.
\end{resumo}
% ---

% ---
% inserir lista de ilustrações
% ---
%\pdfbookmark[0]{\listfigurename}{lof}
%\listoffigures*
%\cleardoublepage
% ---

% ---
% inserir lista de tabelas
% ---
%\pdfbookmark[0]{\listtablename}{lot}
%\listoftables*
%\cleardoublepage
% ---

% ---
% inserir lista de abreviaturas e siglas
% ---
%\begin{siglas}
%  \item[IP] Internet Protocol
%  \item[TCP] Transmission Control Protocol
%  \item[UDP] User Datagram Protocol
%\end{siglas}
% ---

% ---
% inserir o sumario
% ---
\pdfbookmark[0]{\contentsname}{toc}
\tableofcontents*
\cleardoublepage
% ---

% ----------------------------------------------------------
% ELEMENTOS TEXTUAIS
% ----------------------------------------------------------
\textual

\makeatletter
\renewcommand{\chapter}{\@gobbletwo}
\makeatother

\section{Introdução}
\label{sec:introducao}
Tanenbaum e Bos (2016) apontam que um sistema operacional é um dispositivo de \textit{software} cuja função é tornar o computador mais simples e limpo, além de ser responsável pelo gerenciamento dos recursos da máquina.

Ainda nos termos de Tanenbaum e Bos (2016), a maioria dos computadores possui dois modos de operação: modo núcleo (\textit{kernel}) e modo usuário. O sistema operacional, considerado pelos autores a peça mais fundamental do \textit{software}, opera em modo núcleo, o que garante acesso total ao \textit{hardware}, possibilitando a execução de qualquer instrução suportada pela máquina utilizada.

Este trabalho está estruturado da seguinte forma: após a introdução serão apresentados os objetivos do trabalho; em seguida, uma breve fundamentação essencial para o pleno entendimento do trabalho será apresentada; após isso, serão apresentados os materiais utilizados para a realização do trabalho, na seção 4; na seção 5, serão apresentadas e explicadas as etapas realizadas durante esta atividade; a seguir, na seção 6 e 7, os resultados serão discutidos e será apresentado as conclusões do trabalho, respectivamente.



\section{Objetivos}
\label{sec:objetivos}

Este trabalho visa instalar a distribuição GNU/Linux Debian em um ambiente virtual, compilar e instalar o núcleo Linux, e explicar detalhadamente cada um dos comandos utilizados durante todo o procedimento.

\section{Fundamentação}
\label{sec:fundamentacao}


\section{Materiais}
\label{sec:materiais}

\begin{itemize}
  \item Especificações do computador utilizado:
  \begin{itemize}
    \item Modelo: Notebook Lenovo Thinkpad E14
    \item CPU: AMD Ryzen $5$-$3500$U
    \item Memória Principal: $8$GB RAM
    \item Memória Secundária: SSD $256$ NVME
    \item Sistema Operacional: Fedora $40$
  \end{itemize}
  \item Hipervisor: VirtualBox $7.1.2$
  \item Sistema Operacional utilizado no Hipervisor: GNU/Linux Debian $12.7$
  \item Núcleo: Linux $6.10.11$
\end{itemize}

\section{Procedimentos e Resultados}
\label{sec:procedimentos}

Nesta seção, serão detalhados os procedimentos executados para a instalação da distribuição GNU/Linux Debian em um ambiente virtual, utilizando o hipervisor VirtualBox. Serão abordadas as configurações iniciais do hipervisor, a instalação do sistema operacional, verificações essenciais e os comandos utilizados para garantir o funcionamento correto do ambiente. Além disso, será descrito o processo de compilação do núcleo Linux, incluindo as etapas necessárias para a personalização e instalação da nova versão. 

\subsection{Configurações do Hipervisor}
\label{subsec:hipervisor}

Os passos iniciais se deram por meio da instalação do Oracle VM VirtualBox e também da obtenção da imagem do Debian que será usada para a instalação. 

\begin{figure}[H]
  \centering
  \includegraphics[scale=0.7]{figuras/vm.png}
  \caption{Configurações do hipervisor.}
  \label{fig:vm}
\end{figure}

A Figura~\ref{fig:vm} ilustra as principais configurações utilizadas para a instalação do sistema operacional. Para a memória primária, foi selecionado um total de 4 GB e 4 processadores, enquanto a memória secundária foi alocada com 50 GB para uso da máquina virtual.

Além das configurações padrão de processadores e memória, foram incluídas duas placas de rede. O adaptador 1 está configurado como \textit{Network Address Translation} (NAT), permitindo acesso à \textit{internet}, enquanto o adaptador 2 está configurado para uso interno. Essa configuração foi escolhida porque o acesso à máquina será realizado por meio do \textit{Secure Shell Connection} (SSH).

\subsection{Instalação do Sistema Operacional}
\label{subsec:instalacao}

Os primeiros passos durante a instalação envolveram a seleção do idioma. Para facilitar a referência a manuais durante o processo, optou-se pelo inglês. Após essa configuração, o Brasil foi escolhido como localização, permitindo que o sistema operacional definisse o \textit{locale} a ser utilizado para formatações específicas relacionadas à região do usuário~\cite{kerris2010}. Além disso, o relógio do sistema foi ajustado para o fuso horário de São Paulo, e o teclado foi configurado para a disposição adequada da região.

Após as configurações de região, foram criados um usuário comum e um usuário \textit{root}. Na Figura~\ref{fig:root} temos a descrição do \textit{root} como administrador do sistema, possuindo privilégios amplos e acesso total a todos os recursos e configurações do sistema.

\begin{figure}[H]
  \centering
  \includegraphics[scale=0.7]{figuras/root.png}
  \caption{Criação do usuário \textit{root}.}
  \label{fig:root}
\end{figure}

Na Figura~\ref{fig:partition}, estão apresentadas as configurações de particionamento e pontos de montagem do disco. O particionamento permite a divisão entre partições primárias e lógicas, sendo que cada uma delas é derivada do \textit{Master Boot Record} (MBR). De maneira simplificada, as partições primárias são inicializáveis, enquanto as partições lógicas existem apenas como uma forma de contornar algumas limitações existentes na divisão das partições~\cite{archPartition}.

A memória secundária possui 50 GB de armazenamento, o que possibilitou que as partições fossem dispostas da seguinte maneira:

\begin{itemize}
    \item \textit{/}: É a raiz do sistema de arquivos, servindo como ponto de montagem para todos os outros dispositivos e sistemas de arquivos~\cite{tldpRootDir}. Por ser o local onde o sistema operacional inicia e onde estão armazenados arquivos críticos, foi alocado um tamanho de 4 GB para esta partição.
    
    \item \textit{/usr}: Este diretório é responsável por armazenar programas e dados de usuários~\cite{tldpUsrDir}. Por conter a maior quantidade de dados e ser amplamente utilizado durante a compilação, foi escolhido um tamanho de 5 GB para esta partição.
    
    \item \textit{/var}: Armazena dados variáveis, como logs e outros arquivos que podem mudar com frequência~\cite{tldpVarDir}. Por não ser uma área crítica para o funcionamento do sistema, neste trabalho em específico, foi definido um tamanho de apenas 2 GB.
    
    \item \textit{swap}: Esta partição é utilizada para armazenar páginas de memória que estavam na RAM, funcionando como uma extensão da memória virtual do sistema~\cite{archSwap}. Devido à quantidade de RAM já alocada durante a configuração do hipervisor, optou-se por um tamanho de apenas 500 MB para a partição \textit{swap}.
    
    \item \textit{/home}: Este diretório armazena os arquivos de configuração pessoal de cada usuário do sistema~\cite{tldpHomeDir}. Como é uma área frequentemente utilizada pelos usuários, foi definido um tamanho maior do que o das demais partições, totalizando 10 GB.
\end{itemize}

\begin{figure}[H]
  \centering
  \includegraphics[scale=0.7]{figuras/partition.png}
  \caption{Partições do sistema.}
  \label{fig:partition}
\end{figure}

\subsection{Verificações e Comandos}
\label{subsec:verificacao}

Após o processo de instalação, alguns comandos foram realizados para verificação e teste do sistema instalado.

\subsubsection{ps aux}
A Figura~\ref{fig:ps}, mostra a saída do comando \textit{ps} utilizado para a verificação dos processos em execução no sistema~\cite{manPs}. A opção ``aux'', é uma divisão de três parâmetros que alteram o comando da seguinte maneira:

\begin{itemize}
    \item 'a': Mostra os processo de todos os usuários.
    
    \item 'u': Altera o formato da saída do programa.
    
    \item 'x': Inclui processos de segundo plano, como \textit{daemons}.
\end{itemize}

\begin{figure}[H]
  \centering
  \includegraphics[scale=0.37]{figuras/ps_aux.png}
  \caption{Saída do comando \textit{ps aux}.}
  \label{fig:ps}
\end{figure}

\subsubsection{df}
A Figura~\ref{fig:df} apresenta a saída do comando \textit{df}, utilizado para verificar o espaço utilizado pelos sistemas de arquivos no sistema~\cite{manDf}. A opção ``-h'' exibe a saída de maneira mais legível, facilitando a interpretação dos dados.

\begin{figure}[H]
  \centering
  \includegraphics[scale=0.5]{figuras/df.png}
  \caption{Saída do comando \textit{df -h}.}
  \label{fig:df}
\end{figure}

\subsubsection{free}
A Figura~\ref{fig:free} apresenta a saída do comando \textit{free}, utilizado para verificar a utilização da memória no sistema~\cite{manFree}. A opção ``-b'' exibe os valores de memória em bytes, proporcionando uma visão detalhada da quantidade exata de memória disponível e utilizada.

\begin{figure}[H]
  \centering
  \includegraphics[scale=0.5]{figuras/free.png}
  \caption{Saída do comando \textit{free -b}.}
  \label{fig:free}
\end{figure}

\subsubsection{cat /proc/meminfo}
A Figura~\ref{fig:meminfo} apresenta a saída do comando \textit{cat /proc/meminfo}. O comando \textit{cat} é utilizado para exibir o conteúdo de arquivos de texto no terminal~\cite{manCat}. Neste caso, \textit{/proc/meminfo} é um arquivo do sistema que fornece dados em tempo real sobre a utilização da memória física e da \textit{swap}~\cite{tldpProc}.

\begin{figure}[H]
  \centering
  \includegraphics[scale=0.5]{figuras/proc.png}
  \caption{Saída do comando \textit{cat /proc/meminfo}.}
  \label{fig:meminfo}
\end{figure}

\subsubsection{ip}

A Figura~\ref{fig:ip} apresenta as saídas dos comandos \textit{ip -c address show} e \textit{ip -c route}. O comando \textit{ip} é parte das ferramentas do sistema Linux utilizadas para exibir e configurar informações de rede~\cite{manIP}. O parâmetro ''-c`` é utilizado para adicionar cores às saídas, facilitando a leitura das informações.

O comando \textit{ip -c address show} fornece dados importantes, como o endereço Internet Protocol (IP), máscara de rede, \textit{status} das interfaces e outras configurações relevantes.

O comando \textit{ip -c route} exibe a tabela de roteamento.

\begin{figure}[H]
  \centering
  \includegraphics[scale=0.5]{figuras/ip.png}
  \caption{Saídas dos comandos \textit{ip -c address show} e \textit{ip -c route}.}
  \label{fig:ip}
\end{figure}

\subsubsection{cat /etc/resolv.conf}
A Figura~\ref{fig:resolv} mostra a saída do comando \textit{cat /etc/resolv.conf}, que exibe o conteúdo do arquivo responsável pela configuração dos servidores \textit{Domain Name Server} (DNS) no sistema~\cite{manCat}. Esse arquivo contém informações sobre quais servidores DNS o sistema deve consultar para resolver nomes de domínio em endereços IP. A configuração correta deste arquivo é essencial para garantir que o sistema consiga acessar serviços na internet de forma eficaz. 

\begin{figure}[H]
  \centering
  \includegraphics[scale=0.37]{figuras/resolv.png}
  \caption{Conteúdo do arquivo \textit{/etc/resolv.conf}.}
  \label{fig:resolv}
\end{figure}

\subsubsection{cat /etc/network/interfaces}
A Figura~\ref{fig:interfaces} apresenta a saída do comando \textit{cat /etc/network/interfaces}, que exibe as configurações de rede do sistema~\cite{manCat}. Este arquivo define as interfaces de rede disponíveis e suas configurações, como endereços IP, máscara de sub-rede e gateways~\cite{manResolvConf}.

\begin{figure}[H]
  \centering
  \includegraphics[scale=0.37]{figuras/interfaces.png}
  \caption{Conteúdo do arquivo \textit{/etc/network/interfaces}.}
  \label{fig:interfaces}
\end{figure}

\subsubsection{ping}
A Figura~\ref{fig:ping} mostra a saída do comando \textit{ping google.com}, utilizado para verificar a conectividade com o host especificado~\cite{manPing}. O comando envia pacotes ICMP Echo Request para o servidor do Google e mede o tempo de resposta, permitindo avaliar a latência da conexão.

\begin{figure}[H]
  \centering
  \includegraphics[scale=0.37]{figuras/ping.png}
  \caption{Saída do comando \textit{ping google.com}.}
  \label{fig:ping}
\end{figure}

\subsubsection{Configuração de Repositórios}
O arquivo \textit{sources.list} é um componente essencial do gerenciador de pacotes \textit{apt} no Debian~\cite{ubuntuApt}. Neste arquivo é definido os repositórios de onde o sistema obtém os pacotes de software e suas respectivas atualizações. Cada linha do arquivo especifica uma fonte de pacotes, que pode ser um repositório oficial ou um repositório de terceiros.

O \textit{security source list} é uma configuração específica que permite ao sistema acessar repositórios dedicados a atualizações de segurança. Esses repositórios contêm pacotes que corrigem vulnerabilidades e problemas de segurança, sendo fundamental para manter o sistema protegido contra ameaças. 

Como apresentado na Figura~\ref{fig:sources} o Debian 12.7 utilizado neste trabalho, já está previamente configurado com as entradas necessárias para acessar os repositórios principais e de segurança.

\begin{figure}[H]
  \centering
  \includegraphics[scale=0.3]{figuras/source_list.png}
  \caption{Conteúdo do arquivo \textit{/etc/apt/sources.list}.}
  \label{fig:sources}
\end{figure}

\subsubsection{uname}
A Figura~\ref{fig:uname} mostra a saída do comando \textit{uname}, utilizado para exibir informações sobre o sistema operacional em execução~\cite{manUname}. Este comando tem como saída detalhes sobre o sistema, como o nome do kernel, a versão do kernel, a arquitetura do processador e o nome do host. O parâmetro ''-a`` é usado para que a saída mostre todas as informações disponíveis.

\begin{figure}[H]
  \centering
  \includegraphics[scale=0.43]{figuras/uname.png}
  \caption{Saída do comando \textit{uname -a}.}
  \label{fig:uname}
\end{figure}

\subsubsection{Comandos Gerais}
Existem comandos fundamentais para a interação com o sistema operacional e a realização de diversas operações. Abaixo, apresentamos uma lista de comumente usados:

\begin{itemize}
    \item \texttt{clear}: Restaura o terminal a um estado inicial, "limpando" todas as saídas anteriores e comandos executados~\cite{manClear}.

    \item \texttt{ls -l}: Exibe uma lista detalhada dos arquivos e diretórios no diretório especificado. Caso nenhum diretório seja indicado, a listagem será feita do diretório atual~\cite{manLs}.

    \item \texttt{cd}: Altera o diretório de trabalho atual para o diretório especificado no parâmetro, permitindo a navegação pelo sistema de arquivos~\cite{manCd}.

    \item \texttt{cat}: Concatena e exibe o conteúdo de um ou mais arquivos na saída padrão~\cite{manCat}.

    \item \texttt{rm}: Remove permanentemente o arquivo especificado do sistema~\cite{manRm}.

    \item \texttt{nano}: Editor de texto em linha de comando que permite a criação e modificação de arquivos de texto diretamente no terminal~\cite{archNano}.

    \item \texttt{cp}: Copia um arquivo de origem para o destino especificado~\cite{manCp}.

    \item \texttt{grep}: Busca e exibe as linhas de um arquivo que correspondem a um padrão especificado pelo usuário~\cite{manGrep}.

    \item \texttt{head}: Exibe as primeiras linhas de um arquivo, por padrão somente as 10 primeiras linhas são exibidas~\cite{manHead}. 

    \item \texttt{tail}: Mostra as últimas linhas de um arquivo. Similar ao comando \texttt{head}, também exibe 10 linhas por padrão~\cite{manTail}.

    \item \texttt{mv}: Move ou renomeia arquivos e ~\cite{manMv}. 
\end{itemize}

A Figura~\ref{fig:comandos} e a Figura~\ref{fig:comandos2} ilustram o uso desses comandos.

\begin{figure}[H]
  \centering
  \includegraphics[scale=0.5]{figuras/commons.png}
  \caption{Exemplo de comandos gerais do Linux em um terminal.}
  \label{fig:comandos}
\end{figure}

\begin{figure}[H]
  \centering
  \includegraphics[scale=0.5]{figuras/commons2.png}
  \caption{Exemplo de comandos gerais do Linux em um terminal.}
  \label{fig:comandos2}
\end{figure}

\subsection{Compilação do Núcleo}
\label{subsec:compilacao}

Para o processo de compilação do núcleo Linux, é essencial instalar uma série de programas e bibliotecas necessárias. O comando utilizado para essa instalação é ``apt install build-essentials bc dwarves bison flex gnupg libncurses-dev libelf-dev libssl-dev wget''. Este comando instala um conjunto abrangente de pacotes no sistema, cada um com funções específicas que contribuem para a compilação e configuração do núcleo.

\begin{itemize}
    \item \texttt{build-essentials}: Um conjunto de pacotes essenciais para construção de pacotes no Debian~\cite{buildEssentials}.

    \item \texttt{bc}: Calculadora de linha de comando~\cite{manBc}.

    \item \texttt{dwarves}: Conjunto de ferramentas depuração para arquivos no formato Executable and Linkable Format(ELF)~\cite{debianDwarf}.

    \item \texttt{bison}: Gerador de analisador sintático de uso geral~\cite{debianBison}. 

    \item \texttt{flex}: Gerador de analisadores léxicos~\cite{debianFlex}.
    
    \item \texttt{gnupg}: Ferramenta para encriptação de dados e geração de assinaturas digitais~\cite{debianGnupg}

    \item \texttt{libncurses-dev}: Biblioteca de manipulação de caracteres em terminais~\cite{debianLibncurses}.

    \item \texttt{libelf-dev}: Biblioteca para leitura e escrita de arquivos ELF~\cite{debianLibelf}.
    
    \item \texttt{libssl-dev}: Parte do projeto OpenSSL para implementação do protocolo Secure Socket Layer(SSL) e Transport Layer Security (TLS)~\cite{debianLibssl}

    \item \texttt{wget}: Utilitário para obter arquivos usando Hypertext Transfer Protocol (HTTP) ou File Transfer Protocol (FTP)~\cite{debianWget}.
\end{itemize}

Durante a realização deste trabalho, a versão mais recente do núcleo Linux disponível era a 6.10.11. A obtenção dessa versão foi realizada por meio do comando ``wget https://cdn.kernel.org/pub/linux/kernel/v6.x/linux-6.10.11.tar.xz''. O código-fonte é fornecido em um arquivo comprimido, o que requer sua extração. Isso pode ser feito com o comando ``tar -xf linux-6.10.11.tar.xz -C /usr/src/linux-6.10.11/''. A Figura~\ref{wget} ilustra este processo.

\begin{figure}[H]
  \centering
  \includegraphics[scale=0.3]{figuras/wget.png}
  \caption{Obtenção do núcleo e extração do código-fonte.}
  \label{fig:wget}
\end{figure}

A compilação do núcleo Linux é baseada em um arquivo de configuração, que no diretório de compilação é denominado por ``.config''. No Debian, esse arquivo pode ser encontrado no diretório ``/boot'', onde estão armazenados diversos parâmetros que serão utilizados durante o processo de compilação. Antes de iniciar a compilação, é necessário obter esse arquivo de configuração e copiá-lo para o diretório onde a compilação será realizada, que neste trabalho é o diretório ``/usr/src/linux-6.10.11/''. Para isso, foi utilizado o comando ``cp /boot/config-\$(uname -r) .config''.

Após essa configuração inicial, o comando ``make localmodconfig'' é executado. Esse comando ajusta o arquivo de configuração de modo que apenas os módulos já carregados no sistema sejam considerados durante a compilação~\cite{kernelConfig}.

Para as configurações de módulo o comando ``make menuconfig'' é usado, com ele uma interface é apresentada para o usuário de modo que o mesmo possa escolher algumas opções que serão ou não compiladas com o núcleo. Na Figura~\ref{fig:usb} temos o uso deste menu para a habilitação do módulo para Universal Serial Bus (USB) \textit{mass storage}, o que basicamente habilita o suporte para \textit{pen drive}, alguns tipos de dispositivos de disco rígido via USB e afins. Já a Figura~\ref{fig:ntfs} utiliza o menu para habilitar o NT File System (NTFS) um sistema de arquivos utilizado no Windows.

\begin{figure}[H]
  \centering
  \includegraphics[scale=0.3]{figuras/usb.png}
  \caption{Habilitando módulo para USB \textit{mass storage}.}
  \label{fig:usb}
\end{figure}

\begin{figure}[H]
  \centering
  \includegraphics[scale=0.3]{figuras/ntfs.png}
  \caption{Habilitando módulo para sistema de arquivos NTFS.}
  \label{fig:ntfs}
\end{figure}

O uso deste menu não se limita apenas à habilitação de módulos importantes, mas também permite a desabilitação de módulos que não serão utilizados. A Figura~\ref{fig:macintosh} ilustra a configuração necessária para desabilitar \textit{drivers} de dispositivos específicos para computadores Macintosh.

\begin{figure}[H]
  \centering
  \includegraphics[scale=0.3]{figuras/macintosh.png}
  \caption{Desabilitando módulos de \textit{driver} para computadores Macintosh.}
  \label{fig:macintosh}
\end{figure}

Após as configurações no núcleo a compilação pode ser realizada. Na compilação do núcleo o comando ``make -j2 LOCALVERSION=-utfpr'' foi utilizado, onde ``'-j2' especifica a quantidade de núcleos utilizados durante a compilação e ``LOCALVERSION=-utfpr'' é um parametro opcional que especifica um nome que será concatenado ao nome do núcleo.

Após as configurações no núcleo, o processo de compilação pode ser iniciado. Para isso, foi utilizado o comando ``make -j2 LOCALVERSION=-utfpr''. Nesse comando, ``-j2'' especifica o número de núcleos a serem utilizados durante a compilação, otimizando o tempo de execução. O parâmetro opcional ``LOCALVERSION=-utfpr'' adiciona um sufixo ao nome do núcleo, facilitando a identificação da versão compilada.

Uma vez concluída a compilação, o comando ``make modules\_install'' deve ser utilizado para instalar os módulos do núcleo compilado. Em seguida, o comando ``make install'' instala o núcleo em si. Vale ressaltar que o novo núcleo só será utilizado após a reinicialização da máquina. A Figura~\ref{fig:uname_pos} mostra a saída do comando ``uname -a'' após a reinicialização do comando.

\begin{figure}[H]
  \centering
  \includegraphics[scale=0.3]{figuras/uname_pos.png}
  \caption{Saída do comando ``uname -a'' após a compilação do núcleo.}
  \label{fig:uname_pos}
\end{figure}

A Figura~\ref{fig:boot} mostra a saída do comando ``du -sh /lib/modules/6.10.11-utfpr/'' que retorna o espaço em disco ocupado pelos módulos do núcleo, e também a saída do comando ``ls -lh /boot'' que tem como saída os arquivos essenciais para o funcionamento do núcleo, nele encontramos duas versões do núcleo sobre o nome ``vmlinuz'', as configurações de compilação do mesmo sobre o nome ``config'' o ``initrd'' usado para iniciar o sistema operacional e por fim ``System.map'' que armazena alguns símbolos do núcleo para facilitar a depuração~\ref{linuxKernel}.

\begin{figure}[H]
  \centering
  \includegraphics[scale=0.3]{figuras/boot.png}
  \caption{Obtenção do núcleo e extração do código-fonte.}
  \label{fig:boot}
\end{figure}



\section{Discussão dos Resultados}
\label{sec:discussao}

\section{Conclusões}
\label{sec:conclusoes}

% ----------------------------------------------------------
% ELEMENTOS PÓS-TEXTUAIS
% ----------------------------------------------------------
\postextual
% ----------------------------------------------------------
% Referências bibliográficas
% ----------------------------------------------------------
\renewcommand{\bibsection}{%
\section{\bibname}
\bibmark
%\ifnobibintoc\else
%\phantomsection
%\addcontentsline{toc}{section}{\bibname}
%\fi
\prebibhook}

\bibliography{abntex2-modelo-references}



% ----------------------------------------------------------
% Apêndices
% ----------------------------------------------------------

% ---
% Inicia os apêndices
% ---
\begin{apendicesenv}

% ----------------------------------------------------------
\section*{Apêndice A - Nome do Apêndice}
\addcontentsline{toc}{section}{Apêndice A - Nome do Apêndice}
% ----------------------------------------------------------

\end{apendicesenv}
% ---


% ----------------------------------------------------------
% Anexos
% ----------------------------------------------------------

% ---
% Inicia os anexos
% ---
\begin{anexosenv}

% ---
\section*{Anexo A - Nome do Anexo}
\addcontentsline{toc}{section}{Anexo A - Nome do Anexo}
% ---
\end{anexosenv}


\end{document}
